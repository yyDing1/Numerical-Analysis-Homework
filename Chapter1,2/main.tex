\documentclass{article}
\usepackage[utf8]{inputenc}
\usepackage{ctex}
\usepackage{setspace}
\usepackage[margin=1in]{geometry}
\setlength{\baselineskip}{30pt}


\title{第一次作业}
\author{1929401206 丁誉洋}
\date{2022.4.4}

\begin{document}

\maketitle


\noindent\textbf{题目 1}: 对于方程 $f(x) = x^3 - 7.84x - 7.68 = 0$,取步长 $h = 1$ 搜索正根所在区间,并对求出的含根区间估计用二分法求正根时所需的步数(精度要求为 $\epsilon=10^{-4}$)

\noindent 解:

因为 $f(0) = -7.68, f(1) = -14.52, f(2) = -15.36, f(3) = -4.2, f(4) = 24.96$
故可取初始区间 $[a_0, b_0] = [3, 4]$

此时,$K = [\frac{\lg(4 - 3) + 4}{\lg(2)}] = [13.29] = 13$

所需步数为 $13$ \\

\noindent\textbf{题目 2}: 用二分法求解方程 $f(x) = x^2 - 2\sin x - 2 = 0$ 在 $[1.5, 2]$ 内的根(精度要求 $\epsilon = 10^{-3}$)。

\noindent 解:

步数 $K = [\frac{\lg(2 - 1.5) + 3}{\lg(2)}] = [8.97] = 8$ \\

\begin{tabular}{|c|l|l|l|c|}
    \hline
    $k$ & $a_k$ & $b_k$ & $x_k$ & $f(x_k)$ 的符号 \\
    \hline
    0 & 1.5 & 2.0 & 1.75 & - \\
    1 & 1.75 & 2.0 & 1.875 & - \\
    2 & 1.875 & 2.0 & 1.9375 & - \\
    3 & 1.9375 & 2.0 & 1.9688 & + \\
    4 & 1.9375 & 1.9688 & 1.9531 & - \\
    5 & 1.9531 & 1.9688 & 1.9609 & - \\
    6 & 1.9609 & 1.9688 & 1.9648 & + \\
    7 & 1.9609 & 1.9648 & 1.9629 & + \\
    8 & 1.9609 & 1.9629 & 1.9619 & + \\
    \hline
\end{tabular} \\

所以 $\widetilde{x} = \frac{(a_{8} + b_{8})}{2} = 1.962$ 即为所求近似值 \\


\noindent\textbf{题目 3} 为求方程 $x^3 - x^2 - 1 = 0$ 在 $x=1.5$ 附近的一个根,将方程改写为下列等价形式,并建立相应的迭代格式。

(1) $x = 1 + \frac{1}{x^2}$,迭代格式为 $x_{k + 1} = 1 + \frac{1}{x_k^2}$

(2) $x^3 = 1 + x^2$,迭代格式为 $x_{k + 1} = \sqrt[3]{1 + x_k^2}$

(3) $x^2 = \frac{1}{x - 1}$,迭代格式为 $x_{k + 1} = \frac{1}{\sqrt{x_k - 1}}$

讨论每种形式的收敛性,并用格式 (2) 求出精度为 $10^{-2}$ 的根的近似值。


\noindent 解:

\noindent (1) 

因为 $g(x) = 1 + \frac{1}{x^2}, g'(x) = -\frac{2}{x^3}, |g'(1.5)| = 0.59 < 1$

所以取 $x_0 = 1.5$ 该迭代形式收敛

\noindent (2)

因为 $g(x) = \sqrt[3]{1 + x_k^2}, g'(x) = \frac{2}{3}x(1 + x^2)^{-\frac{2}{3}}, |g'(1.5)| = 0.46 < 1$

所以取 $x_0 = 1.5$ 该迭代形式收敛,迭代结果如下

\begin{center}
\begin{tabular}{|c|c|}
    \hline
    $k$ & $x_k$ \\
    \hline
    1 & 1.481 \\
    2 & 1.473 \\
    3 & 1.469 \\
    4 & 1.467 \\
    5 & 1.466 \\
    6 & 1.465 \\
    7 & 1.465 \\
    \hline
\end{tabular}
\end{center}

所以近似值为 $1.47$ \\

\noindent (3)

因为 $g(x) = \frac{1}{\sqrt{x - 1}}, g'(x) = -\frac{1}{2}(x - 1)^{-\frac{3}{2}}, |g'(1.5)| = 1.42 > 1$

所以取 $x_0 = 1.5$ 该迭代形式不收敛 \\

\noindent\textbf{问题 4} 给定代数方程 $f(x) = x^3 + 2x - 3 = 0$

(1) 取 $x_0 = 0$ 用牛顿迭代法球其正根 $x^* = 1$ 的近似值(精度要求为 $\epsilon=10^{-2}$);

\noindent 解@

\noindent (1)

$f(x) = x^3 + 2x - 3, f'(x) = 3x^2 + 2$,所以该方程的牛顿迭代公式为 $x_{k + 1} = x_k - \frac{x_k^3 + 2x_k - 3}{3x_k^2 + 2}$

取 $x_0 = 0$,计算结果如下

\begin{center}
\begin{tabular}{|c|c|}
    \hline
    $k$ & $x_k$ \\
    \hline
    1 & 1.5 \\
    2 & 1.114 \\
    3 & 1.007 \\
    4 & 1.000 \\
    5 & 1.000 \\
    \hline
\end{tabular}
\end{center}

所以近似值为 $1.00$ \\

\noindent\textbf{问题 5} 给定代数方程 $x^2 - 0.1x - 3.06 = 0$,取 $x_{-1} = 1, x_0 = 2$,用弦截法求解正根。$\epsilon=10^{-3}$

\noindent 解:

对于该方程的弦截法公式为

$x_{k + 1} = x_k - \frac{x_k^2 - 0.1x_k - 3.06}{(x_k^2 - 0.1x_k - 3.06) - (x_{k - 1}^2 - 0.1x_{k - 1} - 3.06)}(x_k - x_{k - 1})$

取$x_{-1} = 1, x_0 = 2$,结果见下表

\begin{center}
\begin{tabular}{|c|c|}
    \hline
    $k$ & $x_k$ \\
    \hline
    1 & 1.7448 \\
    2 & 1.7970 \\
    3 & 1.8000 \\
    4 & 1.8000 \\
    \hline
\end{tabular}
\end{center}

所以近似值为 $1.800$ \\

\end{document}