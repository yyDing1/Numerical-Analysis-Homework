\documentclass{article}
\usepackage[utf8]{inputenc}
\usepackage{ctex}
\usepackage{setspace}
\usepackage[margin=1in]{geometry}
\usepackage{graphicx}
\usepackage{amsmath}
\usepackage[colorlinks,linkcolor=blue]{hyperref}
\setlength{\baselineskip}{30pt}
\RequirePackage{listings}
\RequirePackage{xcolor}
\definecolor{dkgreen}{rgb}{0,0.6,0}
\definecolor{gray}{rgb}{0.5,0.5,0.5}
\definecolor{mauve}{rgb}{0.58,0,0.82}
\lstset{
	numbers=left,  
	frame=tb,
	aboveskip=3mm,
	belowskip=3mm,
	showstringspaces=false,
	columns=flexible,
	framerule=1pt,
	rulecolor=\color{gray!35},
	backgroundcolor=\color{gray!5},
	basicstyle={\ttfamily},
	numberstyle=\tiny\color{gray},
	keywordstyle=\color{blue},
	commentstyle=\color{dkgreen},
	stringstyle=\color{mauve},
	breaklines=true,
	breakatwhitespace=true,
	tabsize=4,
}

\title{第二次作业}
\author{1929401206 丁誉洋}
\date{2022.5.8}

\begin{document}

\maketitle


\noindent\textbf{题目 1}: 分别用高斯消去法和高斯-若当消去法求解线性方程组

$$
\begin{pmatrix}
    1 & 4 & 2 \\
    1 & 5 & 2 \\
    0 & 1 & 1
\end{pmatrix}
\begin{pmatrix}
    x_1 \\
    x_2 \\
    x_3
\end{pmatrix}
=
\begin{pmatrix}
    2 \\
    3 \\
    1
\end{pmatrix}
$$

\noindent 解:

\noindent (1) 高斯消去法:

$$
(A\vdots b)
=
\begin{pmatrix}
    1 & 4 & 2 & \vdots & 2 \\
    1 & 5 & 2 & \vdots & 3 \\
    0 & 1 & 1 & \vdots & 1
\end{pmatrix}
\longrightarrow
\begin{pmatrix}
    1 & 4 & 2 & \vdots & 2 \\
    0 & 1 & 0 & \vdots & 1 \\
    0 & 1 & 1 & \vdots & 1 \\
\end{pmatrix}
\longrightarrow
\begin{pmatrix}
    1 & 4 & 2 & \vdots & 2 \\
    0 & 1 & 0 & \vdots & 1 \\
    0 & 0 & 1 & \vdots & 0 \\
\end{pmatrix}
\longrightarrow
\begin{pmatrix}
    1 & 4 & 2 & \vdots & 2 \\
    0 & 1 & 0 & \vdots & 1 \\
    0 & 0 & 1 & \vdots & 0 \\
\end{pmatrix}
$$

把原方程等价约化为:
$$
\left\{
\begin{aligned}
    x_1 + 4x_2 + 2x_3 &= 2 \\
    x_2 + 0x_3 &= 1 \\
    x_3 &= 0
\end{aligned}
\right.
$$

据之回代解得:

$$
\left\{
\begin{aligned}
    x_1 &= -2 \\
    x_2 &= 1 \\
    x_3 &= 0
\end{aligned}
\right.
$$

\noindent (2) 高斯-若当消去法

$$
(A\vdots b)
=
\begin{pmatrix}
    1 & 4 & 2 & \vdots & 2 \\
    1 & 5 & 2 & \vdots & 3 \\
    0 & 1 & 1 & \vdots & 1
\end{pmatrix}
\longrightarrow
\begin{pmatrix}
    1 & 4 & 2 & \vdots & 2 \\
    0 & 1 & 0 & \vdots & 1 \\
    0 & 1 & 1 & \vdots & 1 \\
\end{pmatrix}
\longrightarrow
\begin{pmatrix}
    1 & 0 & 2 & \vdots & -2 \\
    0 & 1 & 0 & \vdots & 1 \\
    0 & 0 & 1 & \vdots & 0 \\
\end{pmatrix}
\longrightarrow
\begin{pmatrix}
    1 & 0 & 0 & \vdots & -2 \\
    0 & 1 & 0 & \vdots & 1 \\
    0 & 0 & 1 & \vdots & 0 \\
\end{pmatrix}
$$

所以,解为 
$
x=
\begin{pmatrix}
    -2 \\
    1 \\
    0 \\
\end{pmatrix}
$ \\

\noindent\textbf{题目 2}: 用选列主元的高斯消去法求解线性方程组

$$
\begin{pmatrix}
    1 & 2 & 1 & -2 \\
    2 & 5 & 3 & -2 \\
    -2 & -2 & 3 & 5 \\
    1 & 3 & 2 & -3
\end{pmatrix}
\begin{pmatrix}
    x_1 \\
    x_2 \\
    x_3 \\
    x_4
\end{pmatrix}
=
\begin{pmatrix}
    4 \\
    7 \\
    -1 \\
    0
\end{pmatrix}
$$

\noindent 解:

$$
\begin{aligned}
(A\vdots b)
=&
\begin{pmatrix}
    1 & 2 & 1 & -2 & \vdots & 4 \\
    \textbf{2} & 5 & 3 & -2 & \vdots & 7 \\
    -2 & -2 & 3 & 5 & \vdots & -1 \\
    1 & 3 & 2 & -3 & \vdots & 0
\end{pmatrix}
\longrightarrow
\begin{pmatrix}
    1 & 2.5 & 1.5 & -1 & \vdots & 3.5 \\
    0 & -0.5 & -0.5 & -1 & \vdots & 0.5 \\
    0 & \textbf{3} & 6 & 3 & \vdots & 6 \\
    0 & 0.5 & 0.5 & -2 & \vdots & -3.5
\end{pmatrix}
\longrightarrow
\begin{pmatrix}
    1 & 2.5 & 1.5 & -1 & \vdots & 3.5 \\
    0 & 1 & 2 & 1 & \vdots & 2 \\
    0 & 0 & \textbf{0.5} & -0.5 & \vdots & 1.5 \\
    0 & 0 & -0.5 & -2.5 & \vdots & -4.5
\end{pmatrix}
\\
&\longrightarrow
\begin{pmatrix}
    1 & 2.5 & 1.5 & -1 & \vdots & 3.5 \\
    0 & 1 & 2 & 1 & \vdots & 2 \\
    0 & 0 & 1 & -1 & \vdots & 3 \\
    0 & 0 & 0 & \textbf{-3} & \vdots & -3
\end{pmatrix}
\longrightarrow
\begin{pmatrix}
    1 & 2.5 & 1.5 & -1 & \vdots & 3.5 \\
    0 & 1 & 2 & 1 & \vdots & 2 \\
    0 & 0 & 1 & -1 & \vdots & 3 \\
    0 & 0 & 0 & 1 & \vdots & 1
\end{pmatrix}
\end{aligned}
$$

回代解得:

$$
\left\{
\begin{aligned}
    x_1 &= 16 \\
    x_2 &= -7 \\
    x_3 &= 4 \\
    x_4 &= 1
\end{aligned}
\right.
$$\\

\noindent\textbf{题目 3}: 用选全主元的高斯-若当消去法求解如下线性方程组

$$
\begin{pmatrix}
    2 & 1 & -2 \\
    3 & 1 & -4 \\
    1 & -1 & 2
\end{pmatrix}
\begin{pmatrix}
    x_1 \\
    x_2 \\
    x_3
\end{pmatrix}
=
\begin{pmatrix}
    6 \\
    6 \\
    0
\end{pmatrix}
$$

\noindent 解:

$$
\begin{aligned}
(A\vdots b)
=&
\begin{pmatrix}
    2 & 1 & -2 & \vdots & 6 \\
    3 & 1 & \textbf{-4} & \vdots & 6 \\
    1 & -1 & 2 & \vdots & 0
\end{pmatrix}
\longrightarrow
\begin{pmatrix}
    1 & -0.25 & -0.75 & \vdots & -1.5 \\
    0 & 0.5 & 0.5 & \vdots & 3 \\
    0 & -0.5 & \textbf{2.5} & \vdots & 3
\end{pmatrix}
\\
&\longrightarrow
\begin{pmatrix}
    1 & 0 & -0.4 & \vdots & -0.6 \\
    0 & 1 & -0.2 & \vdots & 1.2 \\
    0 & 0 & \textbf{0.6} & \vdots & 2.4
\end{pmatrix}
\longrightarrow
\begin{pmatrix}
    1 & 0 & 0 & \vdots & 1 \\
    0 & 1 & 0 & \vdots & 2 \\
    0 & 0 & 1 & \vdots & 4
\end{pmatrix}
\end{aligned}
$$

所以解得

$$
\left\{
\begin{aligned}
    x_1 &= \widetilde{x_2} = 2 \\
    x_2 &= \widetilde{x_3} = 4 \\
    x_3 &= \widetilde{x_1} = 1 \\
\end{aligned}
\right.
$$\\

\noindent\textbf{题目 4}: 用克洛特分解法分解以下矩阵 \\


(1)
$
\mathbf{A}
=
\begin{pmatrix}
    1 & 2 & 1 \\
    0 & 2 & 2 \\
    2 & 4 & 5
\end{pmatrix}
$
\quad
(2)
$
\begin{pmatrix}
    1 & -1 & 1 \\
    5 & -4 & 3 \\
    2 & 1 & 1
\end{pmatrix}
$

\noindent 解:

\noindent (1)

$$
\begin{pmatrix}
    1 & 2 & 1 \\
    0 & 2 & 2 \\
    2 & 4 & 5
\end{pmatrix}
\longrightarrow
\begin{pmatrix}
    \textbf{1} & \textbf{2} & \textbf{1} \\
    \textbf{0} & 2 & 2 \\
    \textbf{2} & 4 & 5
\end{pmatrix}
\longrightarrow
\begin{pmatrix}
    \textbf{1} & \textbf{2} & \textbf{1} \\
    \textbf{0} & \textbf{2} & 2 \\
    \textbf{2} & \textbf{0} & 5
\end{pmatrix}
\longrightarrow
\begin{pmatrix}
    \textbf{1} & \textbf{2} & \textbf{1} \\
    \textbf{0} & \textbf{2} & \textbf{1} \\
    \textbf{2} & \textbf{0} & 5
\end{pmatrix}
\longrightarrow
\begin{pmatrix}
    \textbf{1} & \textbf{2} & \textbf{1} \\
    \textbf{0} & \textbf{2} & \textbf{1} \\
    \textbf{2} & \textbf{0} & \textbf{3}
\end{pmatrix}
$$

解得:

$$
L = 
\begin{pmatrix}
    1 &   &   \\
    0 & 2 &   \\
    2 & 0 & 3
\end{pmatrix}, 
R = 
\begin{pmatrix}
    1 & 2 & 1 \\
      & 1 & 1 \\
      &   & 1
\end{pmatrix}
$$

\noindent (2)

$$
\begin{pmatrix}
    1 & -1 & 1 \\
    5 & -4 & 3 \\
    2 & 1 & 1
\end{pmatrix}
\longrightarrow
\begin{pmatrix}
    \textbf{1} & \textbf{-1} & \textbf{1} \\
    \textbf{5} & -4 & 3 \\
    \textbf{2} & 1 & 1
\end{pmatrix}
\longrightarrow
\begin{pmatrix}
    \textbf{1} & \textbf{-1} & \textbf{1} \\
    \textbf{5} & \textbf{1} & 3 \\
    \textbf{2} & \textbf{3} & 1
\end{pmatrix}
\longrightarrow
\begin{pmatrix}
    \textbf{1} & \textbf{-1} & \textbf{1} \\
    \textbf{5} & \textbf{1} & \textbf{-2} \\
    \textbf{2} & \textbf{3} & 1
\end{pmatrix}
\longrightarrow
\begin{pmatrix}
    \textbf{1} & \textbf{-1} & \textbf{1} \\
    \textbf{5} & \textbf{1} & \textbf{-2} \\
    \textbf{2} & \textbf{3} & \textbf{5}
\end{pmatrix}
$$

解得:

$$
L = 
\begin{pmatrix}
    1 &   &   \\
    5 & 1 &   \\
    2 & 3 & 5
\end{pmatrix}, 
R = 
\begin{pmatrix}
    1 & -1 & 1 \\
      & 1 & -2 \\
      &   & 1
\end{pmatrix}
$$ \\


\noindent\textbf{题目 5}: 用克洛特分解法求解线性方程组

$$
\begin{pmatrix}
    1 & 2 & 1 \\
    2 & 2 & 3 \\
    -1 & -3 & 0
\end{pmatrix}
\begin{pmatrix}
    x_1 \\
    x_2 \\
    x_3
\end{pmatrix}
=
\begin{pmatrix}
    0 \\
    3 \\
    2
\end{pmatrix}
$$

\noindent 解:

对 $A$ 进行克洛特分解

$$
\begin{pmatrix}
    1 & 2 & 1 \\
    2 & 2 & 3 \\
    -1 & -3 & 0
\end{pmatrix}
\longrightarrow
\begin{pmatrix}
    \textbf{1} & \textbf{2} & \textbf{1} \\
    \textbf{2} & 2 & 3 \\
    \textbf{-1} & -3 & 0
\end{pmatrix}
\longrightarrow
\begin{pmatrix}
    \textbf{1} & \textbf{2} & \textbf{1} \\
    \textbf{2} & \textbf{-2} & 3 \\
    \textbf{-1} & \textbf{-1} & 0
\end{pmatrix}
\longrightarrow
\begin{pmatrix}
    \textbf{1} & \textbf{2} & \textbf{1} \\
    \textbf{2} & \textbf{-2} & \textbf{-0.5} \\
    \textbf{-1} & \textbf{-1} & 0
\end{pmatrix}
\longrightarrow
\begin{pmatrix}
    \textbf{1} & \textbf{2} & \textbf{1} \\
    \textbf{2} & \textbf{-2} & \textbf{-0.5} \\
    \textbf{-1} & \textbf{-1} & \textbf{0.5}
\end{pmatrix}
$$

所以

$$
L = 
\begin{pmatrix}
    1 &  &  \\
    2 & -2 &  \\
    -1 & -1 & 0.5
\end{pmatrix}, 
R = 
\begin{pmatrix}
    1 & 2 & 1 \\
      & 1 & -0.5 \\
      &   & 1
\end{pmatrix}
$$

由:
$$
\begin{pmatrix}
    1 &  &  \\
    2 & -2 &  \\
    -1 & -1 & 0.5
\end{pmatrix}
\begin{pmatrix}
    y_1 \\
    y_2 \\
    y_3 
\end{pmatrix}
=
\begin{pmatrix}
    0 \\
    3 \\
    2
\end{pmatrix}
$$

解得:
$$
\begin{pmatrix}
    y_1 \\
    y_2 \\
    y_3
\end{pmatrix} 
= 
\begin{pmatrix}
    0 \\
    -1.5 \\
    1
\end{pmatrix}
$$

由:
$$
\begin{pmatrix}
    1 & 2 & 1 \\
      & 1 & -0.5 \\
      &   & 1
\end{pmatrix}
\begin{pmatrix}
    x_1 \\
    x_2 \\
    x_3 
\end{pmatrix}
=
\begin{pmatrix}
    0 \\
    -1.5 \\
    1
\end{pmatrix}
$$

解得:
$$
\begin{pmatrix}
    x_1 \\
    x_2 \\
    x_3
\end{pmatrix} 
= 
\begin{pmatrix}
    1 \\
    -1 \\
    1
\end{pmatrix}
$$

\noindent\textbf{题目 6}: 取初值 $x_1^{(0)} = x_2^{(0)} = x_3^{(0)} = 0.0$,分别用雅可比迭代法与高斯-塞德尔迭代法解线性方程组(精度要求为 $\epsilon=10^{-3}$)

$$
\begin{pmatrix}
    3.0 & 0.15 & -0.09 \\
    0.08 & 4.0 & -0.16 \\
    0.05 & -0.3 & 5.0
\end{pmatrix}
\begin{pmatrix}
    x_1 \\
    x_2 \\
    x_3
\end{pmatrix}
=
\begin{pmatrix}
    6.09 \\
    11.52 \\
    19.20
\end{pmatrix}
$$

\noindent 解:

相应的迭代公式为

1. 雅可比迭代:

$$
\left\{
\begin{aligned}
    x_1^{k + 1} &= 2.03 - 0.05x_2^k + 0.03x_3^k \\
    x_2^{k + 1} &= 2.88 - 0.02x_1^k + 0.04x_3^k \\
    x_3^{k + 1} &= 3.84 - 0.01x_1^k + 0.06x_2^k
\end{aligned}
\right.
$$

迭代过程为

$$
\begin{pmatrix}
    0 \\
    0 \\
    0
\end{pmatrix}
\longrightarrow
\begin{pmatrix}
    2.03 \\
    2.88 \\
    3.84
\end{pmatrix}
\longrightarrow
\begin{pmatrix}
    2.0012 \\
    2.9930 \\
    3.9925
\end{pmatrix}
\longrightarrow
\begin{pmatrix}
    2.0001 \\
    2.9997 \\
    3.9996
\end{pmatrix}
\longrightarrow
\begin{pmatrix}
    2.0000 \\
    3.0000 \\
    4.0000
\end{pmatrix}
\longrightarrow
\begin{pmatrix}
    2.0000 \\
    3.0000 \\
    4.0000
\end{pmatrix}
$$

所以

$$
\begin{pmatrix}
    x_1^4 \\
    x_2^4 \\
    x_3^4
\end{pmatrix}
=
\begin{pmatrix}
    2.000 \\
    3.000 \\
    4.000
\end{pmatrix}
$$

2. 高斯-塞德尔迭代:

$$
\left\{
\begin{aligned}
    x_1^{k + 1} &= 2.03 - 0.05x_2^k + 0.03x_3^k \\
    x_2^{k + 1} &= 2.88 - 0.02x_1^{k + 1} + 0.04x_3^k \\
    x_3^{k + 1} &= 3.84 - 0.01x_1^{k + 1} + 0.06x_2^{k + 1}
\end{aligned}
\right.
$$

迭代过程为

$$
\begin{pmatrix}
    0 \\
    0 \\
    0
\end{pmatrix}
\longrightarrow
\begin{pmatrix}
    2.0300 \\
    2.8394 \\
    2.9901
\end{pmatrix}
\longrightarrow
\begin{pmatrix}
    2.0077 \\
    2.9994 \\
    3.9999
\end{pmatrix}
\longrightarrow
\begin{pmatrix}
    2.0000 \\
    3.0000 \\
    4.0000
\end{pmatrix}
\longrightarrow
\begin{pmatrix}
    2.0000 \\
    3.0000 \\
    4.0000
\end{pmatrix}
$$

所以

$$
\begin{pmatrix}
    x_1^3 \\
    x_2^3 \\
    x_3^3
\end{pmatrix}
=
\begin{pmatrix}
    2.000 \\
    3.000 \\
    4.000
\end{pmatrix}
$$

\noindent\textbf{题目 7}: 利用定理 3.8 对以下线性方程组讨论雅可比迭代法与高斯-塞德尔迭代法的收敛性。

$$
\begin{pmatrix}
    1 & 2 & -2 \\
    1 & 1 & 1 \\
    2 & 2 & 1
\end{pmatrix}
\begin{pmatrix}
    x_1 \\
    x_2 \\
    x_3
\end{pmatrix}
=
\begin{pmatrix}
    -1 \\
    1 \\
    1
\end{pmatrix}
$$

\noindent 解:

$$
M_J = D^{-1}(L + U) = 
\begin{pmatrix}
    1& 0 & 0 \\
    0 & 1 & 0 \\
    0 & 0 & 1
\end{pmatrix}
\begin{pmatrix}
    0 & -2 & 2 \\
    -1 & 0 & -1 \\
    -2 & -2 & 0
\end{pmatrix}
=
\begin{pmatrix}
    0 & -2 & 2 \\
    -1 & 0 & -1 \\
    -2 & -2 & 0
\end{pmatrix}
$$

特征值 $\lambda_1 = \lambda_2 = \lambda_3 = 0$,所以 $\rho(M) = \max\limits_{1\leq i\leq n}|\lambda_i| = 0 < 1$,所以雅可比迭代法收敛

$$
M_{G-S} = (D - L)^{-1}U
=
\begin{pmatrix}
    1 & 0 & 0 \\
    -1 & 1 & 0 \\
    0 & -2 & 1
\end{pmatrix}
\begin{pmatrix}
    0 & -2 & 2 \\
    0 & 0 & -1 \\
    0 & 0 & 0
\end{pmatrix}
=
\begin{pmatrix}
    0 & -2 & 2 \\
    0 & 2 & -3 \\
    0 & 0 & 2
\end{pmatrix}
$$

特征值 $\lambda_1 = 0, \lambda_2 = \lambda_3 = 2$,所以 $\rho(M) = \max\limits_{1\leq i\leq n}|\lambda_i| = 2 > 1$,所以高斯-塞德尔迭代法无法收敛 \\ \\ \\ \\ \\

\noindent 具体代码实现见下一页,项目地址 \href{https://github.com/yyDing1/Numerical-Analysis-Homework}{Numerical Analysis Homework}

\clearpage
此外,在作业的过程中,我将上述方法封装成了一个 \texttt{Matrix} 类,方便计算,以下是对高斯消元法,高斯-若当消元法(包括主元的选择方法),克洛特分解法,雅可比迭代法和高斯-塞德尔迭代法的 \textbf{C++} 实现,复杂度均为课件中所提到的复杂度


\begin{lstlisting}[language=c++, escapeinside=``]
#include<bits/stdc++.h>
using namespace std;

const int N = 5e2 + 5;
double mat[N][N];

struct Matrix {
	double mat[N][N];
	int n;

	void read(void) {
		scanf("%d", &n);
		for (int i = 0; i < n; i++) {
			for (int j = 0; j < n + 1; j++) {
				scanf("%lf", &mat[i][j]);
			}
		}
	}
	void printMat(double mat[N][N], int n, int m) {
		for (int i = 0; i < n; i++) {
			for (int j = 0; j < m; j++) {
				printf("%f ", mat[i][j]);
			}
			printf("\n");
		}
	}
	void printVec(double vec[], int n) {
		for (int i = 0; i < n; i++) {
			printf("%f ", vec[i]);
		}
		printf("\n");
	}

	// pivot_choice_method must be in ["none", "column", "all"]
	void GaussianElimination(bool GaussionJordan=true, string pivot_choice_method="none") {
		double ans[N];
		int row, col, pos[N];
		for (int i = 0; i < n; i++) {
			pos[i] = i;
		}
		for (row = 0, col = 0; row < n && col < n; row++, col++) {
			printf("Stage %d\n", row);
			if (pivot_choice_method == "column") {
				int maxrow = row;
				for (int i = row; i < n; i++) {
					if (fabs(mat[i][col]) > fabs(mat[maxrow][col])) {
						maxrow = i;
					}
				}
				for (int j = 0; j < n + 1; j++){
					swap(mat[row][j], mat[maxrow][j]);
				}
			}
			else if (pivot_choice_method == "all") {
				int maxrow = row, maxcol = col;
				for (int i = row; i < n; i++) {
					for (int j = col; j < n; j++) {
						if (fabs(mat[i][j]) > fabs(mat[maxrow][maxcol])) {
							maxrow = i; maxcol = j;
						}
					}
				}
				for (int j = 0; j < n + 1; j++) {
					swap(mat[row][j], mat[maxrow][j]);
				}
				for (int i = 0; i < n; i++) {
					swap(mat[i][col], mat[i][maxcol]);
				}
				swap(pos[col], pos[maxcol]);
			}
			if (mat[row][col] == 0) {
				printf("Fail!\n");
				return;
			}
			double div = mat[row][col];
			for (int j = col; j < n + 1; j++) {
				mat[row][j] /= div;
			}
			for (int i = GaussionJordan? 0: row + 1; i < n; i++) {
				if (i == row) continue;
				double temp = mat[i][col];
				for (int j = col; j < n + 1; j++) {
					mat[i][j] -= mat[row][j] * temp;
				}
				mat[i][col] = 0;
			}
			printMat(mat, n, n + 1);
		}
		for (int i = n - 1; i >= 0; i--) {
			ans[i] = mat[i][n];
			for (int j = i + 1; j < n; j++) {
				ans[i] -= ans[j] * mat[i][j];
			}
		}
		printf("Result: \n");
		for (int i = 0; i < n; i++) {
			printf("ans%d = ans'%d = %f\n", pos[i], i, ans[i]);
		}
	}
	void CroutSplit(void) {
		double l[N][N], u[N][N], x[N], y[N];
		for (int i = 0; i < n; i++) {
			double sum;
			for (int j = 0; j <= i; j++) {
				sum = 0;
				for (int k = 0; k < j; k++) {
					sum += l[i][k] * u[k][j];
				}
				l[i][j] = mat[i][j] - sum;
			}
			for (int j = i + 1; j < n; j++) {
				sum = 0;
				for (int k = 0; k < i; k++) {
					sum += l[i][k] * u[k][j];
				}
				u[i][j] = (mat[i][j] - sum) / l[i][i];
			}
			u[i][i] = 1;
		}
		printf("L = \n"); printMat(l, n, n);
		printf("U = \n"); printMat(u, n, n);
		for (int k = 0; k < n; k++) {
			double sum = 0;
			for (int i = 0; i < k; i++) {
				sum += l[k][i] * y[i];
			}
			y[k] = (mat[k][n] - sum) / l[k][k];
		}
		printf("y = \n"); printVec(y, n);
		for (int k = n - 1; k >= 0; k--) {
			double sum = 0;
			for (int i = k + 1; i < n; i++) {
				sum += u[k][i] * x[i];
			}
			x[k] = (y[k] - sum) / u[k][k];
		}
		printf("x = \n"); printVec(x, n);
	}

	void JacobiMethod(double x[], int iter) {
		double newx[N];
		for (int iter_num = 1; iter_num <= iter; iter_num++) {
			for (int i = 0; i < n; i++) {
				double sum = 0;
				for (int j = 0; j < n; j++) {
					if (j == i) continue;
					sum += mat[i][j] * x[j];
				}
				newx[i] = (mat[i][n] - sum) / mat[i][i];
			}
			printf("Iter %d: \n", iter_num);
			for (int i = 0; i < n; i++) {
				x[i] = newx[i];
			}
			printVec(x, n);
		}
	}

	void GaussSeidelMethod(double x[], int iter) {
		double newx[N];
		for (int iter_num = 1; iter_num <= iter; iter_num++) {
			for (int i = 0; i < n; i++) {
				double sum = 0;
				for (int j = 0; j < i; j++) {
					sum += mat[i][j] * newx[j];
				}
				for (int j = i + 1; j < n; j++) {
					sum += mat[i][j] * x[j];
				}
				newx[i] = (mat[i][n] - sum) / mat[i][i];
			}
			printf("Iter %d: \n", iter_num);
			for (int i = 0; i < n; i++) {
				x[i] = newx[i];
			}
			printVec(x, n);
		}
	}
}now;

int main(void) {
	now.read();
	now.GaussianElimination(true, "none");
	// now.CroutSplit();
	// double x[N] = {0, 0, 0};
	// now.JacobiMethod(x, 20)
	// now.GaussSeidelMethod(x, 20);
	return 0;
}
\end{lstlisting}


\end{document}