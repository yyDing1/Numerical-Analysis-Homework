\documentclass{article}
\usepackage[utf8]{inputenc}
\usepackage{ctex}
\usepackage{setspace}
\usepackage[margin=1in]{geometry}
\usepackage{graphicx}
\usepackage{amsmath}
\usepackage{amssymb}
\usepackage[colorlinks,linkcolor=blue]{hyperref}
\setlength{\baselineskip}{30pt}
\RequirePackage{listings}
\RequirePackage{xcolor}
\definecolor{dkgreen}{rgb}{0,0.6,0}
\definecolor{gray}{rgb}{0.5,0.5,0.5}
\definecolor{mauve}{rgb}{0.58,0,0.82}
\lstset{
	numbers=left,  
	frame=tb,
	aboveskip=3mm,
	belowskip=3mm,
	showstringspaces=false,
	columns=flexible,
	framerule=1pt,
	rulecolor=\color{gray!35},
	backgroundcolor=\color{gray!5},
	basicstyle={\ttfamily},
	numberstyle=\tiny\color{gray},
	keywordstyle=\color{blue},
	commentstyle=\color{dkgreen},
	stringstyle=\color{mauve},
	breaklines=true,
	breakatwhitespace=true,
	tabsize=4,
}

\title{第三次作业}
\author{1929401206 丁誉洋}
\date{2022.5.27}

\begin{document}

\maketitle

\noindent\textbf{题目 1}: 对以下数据表分别用线性和二次拉格朗日插值多项式求 $y(0.3)$ 的近似值

\begin{center}
    \begin{tabular}{|c|c|c|c|c|c|}
        \hline
        $x$ & -0.1 & 0.1 & 0.2 & 0.4 & 0.9 \\
        \hline
        $y$ & -2 & 1 & 2 & 7 & 14 \\
        \hline
    \end{tabular}
\end{center}

\noindent 解:

\noindent (1) 线性插值

选取 $x_0 = 0.2, x_1 = 0.4$,得到
$$
\begin{aligned}
L_1(x) &= \frac{x - 0.4}{0.2 - 0.4}\times 2 + \frac{x - 0.2}{0.4 - 0.2}\times 7 \\
\therefore y(0.3) &\approx \frac{0.3 - 0.4}{0.2 - 0.4}\times 2 + \frac{0.3 - 0.2}{0.4 - 0.2}\times 7 \\
&\approx 4.5 \\
\end{aligned}
$$

\noindent (2) 二次插值

选取 $x_0 = 0.1, x_1 = 0.2, x_2 = 0.4$,得到
$$
\begin{aligned}
L_2(x) &= \frac{(x - 0.2)(x - 0.4)}{(0.1 - 0.2)(0.1 - 0.4)}\times 1 + \frac{(x - 0.1)(x - 0.4)}{(0.2 - 0.1)(0.2 - 0.4)}\times 2 + \frac{(x - 0.1)(x - 0.2)}{(0.4 - 0.1)(0.4 - 0.2)}\times 7 \\
\therefore y(0.3) &\approx \frac{(0.3 - 0.2)(0.3 - 0.4)}{(0.1 - 0.2)(0.1 - 0.4)}\times 1 + \frac{(0.3 - 0.1)(0.3 - 0.4)}{(0.2 - 0.1)(0.2 - 0.4)}\times 2 + \frac{(0.3 - 0.1)(0.3 - 0.2)}{(0.4 - 0.1)(0.4 - 0.2)}\times 7 \\
&\approx 4.0
\end{aligned}
$$

\noindent\textbf{题目 2}: 给定数据表

\begin{center}
    \begin{tabular}{|c|c|c|c|c|}
        \hline
        $x$ & 0 & 2 & 5 & 8 \\
        \hline
        $f(x)$ & -5 & 15 & 0 & 3 \\
        \hline
    \end{tabular}
\end{center}

(1) 试建立相应的三次拉格朗日插值多项式

\noindent 解:

\noindent (1)

由题,$x_0 = 0, x_1 = 2, x_2 = 5, x_3 = 8$,得到
$$
\begin{aligned}
L_3(x) =& \frac{(x - 2)(x - 5)(x - 8)}{(0 - 2)(0 - 5)(0 - 8)}\times (-5) + \frac{(x - 0)(x - 5)(x - 8)}{(2 - 0)(2 - 5)(2 - 8)}\times 15 \\
&+ \frac{(x - 0)(x - 2)(x - 8)}{(5 - 0)(5 - 2)(5 - 8)}\times 0 + \frac{(x - 0)(x - 2)(x - 5)}{(8 - 0)(8 - 2)(8 - 5)}\times 3 \\
=& \frac{1}{2}(x - 5)(x^2 - 8x + 2)
\end{aligned}
$$

\noindent\textbf{题目 3}: 给定数据表

\begin{center}
    \begin{tabular}{|c|c|c|c|c|}
        \hline
        $x$ & 0 & 3 & 5 & 6 \\
        \hline
        $f(x)$ & 5 & 128 & 430 & 665 \\
        \hline
    \end{tabular}
\end{center}

用三次插值函数求 $f(2)$ 的值

\noindent 解:
$$
\begin{aligned}
N_3(2) &= f[3] + f[3, 0](2 - 3) + f[3, 0, 5](2 - 3)(2 - 0) + f[3, 0, 5, 6](2 - 3)(2 - 0)(2 - 5) \\
&= 128 + 41\times (-1) + 22\times (-1)\times 2 + 1\times (-1)\times 2\times (-3) \\
&= 49 \\
\therefore f(2) &\approx N_3(2) = 49
\end{aligned}
$$

\noindent\textbf{题目 4}: 设 $l_i(x) (i = 0, 1, \ldots, n)$ 为基本拉格朗日插值多项式,节点 $x_0, x_1, ..., x_n$ 互异,证明

$$
\sum\limits_{i = 0}^{n}l_i(x)x_i^k \equiv x^k \quad\quad (k = 0, 1, \ldots, n)
$$

\noindent 解:

等式左边 $\sum\limits_{i = 0}^{n}l_i(x)x_i^k$ 可等价转化为对 $(x_0, x_0^k), (x_1, x_1^k), \ldots, (x_n, x_n^k)$ 的差值结果,

即 $L_n(x) = \sum\limits_{i = 0}^{n}l_i(x)y_i$,

容易看出 $L_n(x) = x^k$ 是一种可行的差值结果

因为 $k\leq n$,根据定理 4.1,即:
在 $n + 1$ 个互异点 $x_0, x_1, \ldots, x_n$ 上满足插值条件 $P(x_i) = y_i \quad (i = 0, 1, \ldots, n)$ 的次数不超过 $n$ 次的插值多项式 $P_n(x)$ 存在且惟一。

所以 $L_n(x) = x^k$ 存在且唯一

所以
$$
\sum\limits_{i = 0}^{n}l_i(x)x_i^k \equiv x^k \quad\quad (k = 0, 1, \ldots, n)
$$

\noindent\textbf{题目 5}: 设 $f(x) = 2x^4 - 4x^2 + 4x - 1$,求

(1) $f[3^0, 3^1, 3^2, 3^3, 3^4]$;

(2) $f[4^1, 4^2, 4^3, 4^4, 4^5, 4^6]$;

(3) $f[0, 1, 2, 3]$。

\noindent 解:

\noindent (1)

由题
$$
\begin{aligned}
\because N_4(x) &= N_3(x) + f[3^0, 3^1, 3^2, 3^3, 3^4](x - 3^0)(x - 3^1)(x - 3^2)(x - 3^3) \\
N_4(x) &= f(x) = 2x^4 - 4x^2 + 4x - 1 \\
\therefore N_3(x) &+ f[3^0, 3^1, 3^2, 3^3, 3^4](x - 3^0)(x - 3^1)(x - 3^2)(x - 3^3) = 2x^4 - 4x^2 + 4x - 1 \\
\end{aligned}
$$

两边 $x^4$ 的系数分别为 $f[3^0, 3^1, 3^2, 3^3, 3^4]$ 和 2

所以 $f[3^0, 3^1, 3^2, 3^3, 3^4] = 2$

\noindent (2)

$N_4(x)$ 为 $4$ 次多项式,其 6 阶均差函数 $f[4^1, 4^2, 4^3, 4^4, 4^5, 4^6]$,因为 $6 > 4$,所以均差为 $0$

即 $f[4^1, 4^2, 4^3, 4^4, 4^5, 4^6] = 0$

\noindent (3)

$$
\begin{aligned}
&f(0) = -1, f(1) = 1, f(2) = 23, f(3) = 137 \\
&f[0, 1, 2] = \frac{f[0, 2] - f[0, 1]}{2 - 1}= \frac{f(2) - f(0)}{2 - 0} - \frac{f(1) - f(0)}{1 - 0} = 12 - 2 = 10 \\
&f[0, 1, 3] = \frac{f[0, 3] - f[0, 1]}{3 - 1} = \frac{1}{2}(\frac{f(3) - f(0)}{3 - 0} - \frac{f(1) - f(0)}{1 - 0}) = \frac{1}{2}(46 - 2) = 22 \\
&f[0, 1, 2, 3] = \frac{f[0, 1, 3] - f[0, 1, 2]}{3 - 2} = 12
\end{aligned}
$$

\noindent\textbf{题目 6}: 试给出以下数据最合理的拟合曲线。

\begin{center}
    \begin{tabular}{|c|c|c|c|c|c|}
        \hline
        $x$ & 0 & 2 & 4 & 6 & 8 \\
        \hline
        $y$ & -0.2 & 10.1 & 19.9 & 30.1 & 40.1 \\
        \hline
    \end{tabular}
\end{center}

\noindent 解:

描点,确定 $m = 1$。

令 $P(x) = a_0 + a_1x$,计算得 

$S_0 = 5, S_1 = 20, S_2 = 120$

$T_0 = 100, T_1 = 601.2$

从而建立法方程组

$$
\begin{pmatrix}
    5 & 20 \\
    20 & 120
\end{pmatrix}
\begin{pmatrix}
    a_0 \\
    a_1
\end{pmatrix}
=
\begin{pmatrix}
    100 \\
    601.2
\end{pmatrix}
$$

解得 $a_0 = -0.12, a_1 = 5.03$

故 $P(x) = -0.12 + 5.03x$

\noindent\textbf{题目 7}: 用最小二乘法求形如 $y = ax + bx^2$ 的多项式,使与下列数据拟合(得数保留四位小数)。

\begin{center}
    \begin{tabular}{|c|c|c|c|c|c|}
        \hline
        $x$ & -3 & -1 & 0 & 2 & 4 \\
        \hline
        $y$ & -8.2 & -9.2 & 0 & 38.1 & 102.1 \\
        \hline
    \end{tabular}
\end{center}

\noindent 解:

$\because S_2 = 30, S_3 = 44, S_4 = 354, T_1 = 518.4, T_2 = 1703$

$\therefore$ 相应的方程组为

$$
\begin{pmatrix}
    30 & 44 \\
    44 & 354
\end{pmatrix}
\begin{pmatrix}
    a \\
    b
\end{pmatrix}
=
\begin{pmatrix}
    518.4 \\
    1703
\end{pmatrix}
$$

解得 $a\approx 12.5036, b\approx 3.2566$

$\therefore$ 所求拟合多项式为 $y = 12.5036x + 3.2566x^2$

\noindent\textbf{题目 8}: 测得单摆振动的振幅随时间 $t$ 变化的数据表如下,试用指数拟合求解衰减变化规律 $y = ae^{bt}$(得数保留三位小数)。

\begin{center}
    \begin{tabular}{|c|c|c|c|c|c|c|c|}
        \hline
        $t$ & 0 & 1 & 2 & 3 & 4 & 5 & 6 \\
        \hline
        $y$ & 9.00 & 4.47 & 2.22 & 1.10 & 0.55 & 0.27 & 0.13 \\
        \hline
    \end{tabular}
\end{center}

\noindent 解:

$y = ae^{bt} \Longrightarrow z = \ln{y} = \ln{a} + bt$

首先根据 $y_i$ 的值算出 $z_i = \ln{y_i}\quad (t = 0, 1, 2, 3, 4, 5, 6)$

\begin{center}
    \begin{tabular}{|c|c|c|c|c|c|c|c|}
        \hline
        $t$ & 0 & 1 & 2 & 3 & 4 & 5 & 6 \\
        \hline
        $z_i = \ln{y}$ & 2.1972 & 1.4974 & 0.7975 & 0.0953 & -0.5978 & -1.3093 & -2.0402 \\
        \hline
    \end{tabular}
\end{center}

令 $z(t) = a_0 + a_1t$ 对数据 $(t_i, z_i)$ 进行拟合

$S_0 = 7, S_1 = 21, S_2 = 91, T_0 = 0.6401, T_1 = -17.8006$

建立法方程组
$$
\begin{pmatrix}
    7 & 21 \\
    21 & 91
\end{pmatrix}
\begin{pmatrix}
    a_0 \\
    a_1
\end{pmatrix}
=
\begin{pmatrix}
    0.6401 \\
    -17.8006
\end{pmatrix}
$$

解出 $a_0 = 2.204, a_1 = -0.704$

所以 $z(t) = 2.204 - 0.704t$,$y(t) = e^{z(t)} = 9.064e^{-0.704t}$

\noindent\textbf{题目 9}: 试求以下超定方程组的最小二乘解(得数保留三位小数)。
$$
\left\{
\begin{aligned}
x_1 - x_2 =& 1 \\
-2x_1 + x_2 =& 2 \\
2x_1 - 2x_2 =& 3 \\
-3x_1 + x_2 =& 4 \\
\end{aligned}
\right.
$$

\noindent 解:

$$
A
=
\begin{pmatrix}
    1 & -1 \\
    -2 & 1 \\
    2 & -2 \\
    -3 & 1
\end{pmatrix}
,
b
=
\begin{pmatrix}
    1 \\
    2 \\
    3 \\
    4
\end{pmatrix}
$$

$$
A^TA
=
\begin{pmatrix}
    18 & -10 \\
    -10 & 7
\end{pmatrix}
,
A^Tb
=
\begin{pmatrix}
    -9 \\
    -1
\end{pmatrix}
$$

所以法方程为
$$
\begin{pmatrix}
    18 & -10 \\
    -10 & 7
\end{pmatrix}
\begin{pmatrix}
    x_1 \\
    x_2
\end{pmatrix}
=
\begin{pmatrix}
    -9 \\
    -1
\end{pmatrix}
$$

解得 $x_1 = -2.808, x_2 = -4.154$

% \noindent 具体代码实现见下一页,项目地址 \href{https://github.com/yyDing1/Numerical-Analysis-Homework}{Numerical Analysis Homework}

\end{document}
