\documentclass{article}
\usepackage[utf8]{inputenc}
\usepackage{ctex}
\usepackage{setspace}
\usepackage[margin=1in]{geometry}
\usepackage{graphicx}
\usepackage{amsmath}
\usepackage{amssymb}
\usepackage[colorlinks,linkcolor=blue]{hyperref}
\setlength{\baselineskip}{30pt}
\RequirePackage{listings}
\RequirePackage{xcolor}
\definecolor{dkgreen}{rgb}{0,0.6,0}
\definecolor{gray}{rgb}{0.5,0.5,0.5}
\definecolor{mauve}{rgb}{0.58,0,0.82}
\lstset{
	numbers=left,  
	frame=tb,
	aboveskip=3mm,
	belowskip=3mm,
	showstringspaces=false,
	columns=flexible,
	framerule=1pt,
	rulecolor=\color{gray!35},
	backgroundcolor=\color{gray!5},
	basicstyle={\ttfamily},
	numberstyle=\tiny\color{gray},
	keywordstyle=\color{blue},
	commentstyle=\color{dkgreen},
	stringstyle=\color{mauve},
	breaklines=true,
	breakatwhitespace=true,
	tabsize=4,
}

\title{第四次作业}
\author{1929401206 丁誉洋}
\date{2022.6.12}
\begin{document}

\maketitle

\noindent\textbf{题目 1}: 把区间 $[0, 2]$ 分为八等分,采用分点上的函数值,分别以复合梯形公式,复合 Simpson 公式和复合 Cotes 公式计算以下定积分。

$\int_0^2(xe^{-x} + 1)dx$

按照 $4$ 位小数计算

\noindent 解:

\begin{center}
\begin{tabular}{|c|c|}
    \hline
    $x$ & $f(x) = xe^{-x} + 1$ \\
    \hline
    0.00 & 1.0000 \\
    \hline
    0.25 & 1.1947 \\
    \hline
    0.50 & 1.3033 \\
    \hline
    0.75 & 1.3543 \\
    \hline
    1.00 & 1.3679 \\
    \hline
    1.25 & 1.3581 \\
    \hline
    1.50 & 1.3347 \\
    \hline
    1.75 & 1.3041 \\
    \hline
    2.00 & 1.2707 \\
    \hline
\end{tabular}
\end{center}

$T_8 = \frac{2}{2}\frac{1}{8}[f(0) + f(2) + 2\sum\limits_{k = 1}^{7}f(\frac{k}{4})] = 2.5881$

$S_4 = \frac{2}{6}\frac{1}{4}[f(0) + f(1) + 2\sum\limits_{k = 1}^3f(\frac{k}{2}) + 4\sum\limits_{k = 1}^4f(\frac{2k - 1}{4})] = 2.5939$

$C_2 = \frac{2}{90}\frac{1}{2}[7(f(0) + f(2)) + 14f(1) + 12(f(\frac{1}{2}) + f(\frac{3}{2})) + 32(f(\frac{1}{4}) + f(\frac{3}{4}) + f(\frac{5}{4}) + f(\frac{7}{4}))] = 2.5940$ \\

\noindent\textbf{题目 2}: 取步长 $h = 0.1$ 用改进欧拉公式求解常微分方程初值问题
$$
\left\{
\begin{aligned}
&y' + xy = 0 \\
&y(0) = 1
\end{aligned}
\right.
$$
在 $x = 0.4$ 处的近似值。按照四位小数计算。


由改进欧拉公式得
$$
\left\{
\begin{aligned}
\Tilde{y}_{i + 1} &= y_i - 0.1\times x_iy_i \\
y_{i + 1} &= y_i - \frac{h}{2}(x_iy_i + x_{i + 1}\Tilde{y}_{i + 1})
\end{aligned}
\right.
$$

% y(0) &= 1 \\
% \Tilde{y}(0.1) &= 1 - 0.1\times 0\times 1 = 1 \\
% \Tilde{y}(0.2) &= 1 - 0.1\times 0.1\times 1 = 0.99 \\
% \Tilde{y}(0.3) &= 0.99 - 0.1\times 0.2\times 0.99 = 0.9702 \\
% \Tilde{y}(0.4) &= 0.9702 - 0.1\times 0.3 \times 0.9702 = 0.9411

所以
$$
\begin{aligned}
y(0) &= 1 \\
\Tilde{y}(0.1) &= 1 - 0.1\times 0\times 1 = 1 \\
y(0.1) &= 1 - 0.05\times (0\times 1 + 0.1\times 1) = 0.995 \\
\Tilde{y}(0.2) &= 0.995 - 0.1\times 0.1\times 0.995 = 0.9851 \\
y(0.2) &= 0.995 - 0.05\times (0.1\times 0.995 + 0.2\times 0.9851) = 0.9802 \\
\Tilde{y}(0.3) &= 0.9802 - 0.1\times 0.2\times 0.9802 = 0.9606 \\
y(0.3) &= 0.9802 - 0.05\times (0.2\times 0.9802 + 0.3\times 0.9606) = 0.9560 \\
\Tilde{y}(0.4) &= 0.9560 - 0.1\times 0.3\times 0.9560 = 0.9273 \\
y(0.4) &= 0.9560 - 0.05\times (0.3\times 0.9560 + 0.4\times 0.9273) = 0.9231
\end{aligned}
$$

近似值为 $y(0.4) = 0.9231$

\end{document}
